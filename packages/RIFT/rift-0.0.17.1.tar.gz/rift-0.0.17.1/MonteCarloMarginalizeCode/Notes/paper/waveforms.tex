\subsection{Intrinsic and extrinsic parameters}

On physical grounds, we group waveform parameters $\vec{\mu}=(\vec{\lambda},\vec{\theta})$ into two classes: the
intrinsic parameters ($\vec{\lambda}$) and extrinsic parameters ($\vec{\theta}$). 
The intrinsic parameters are fundamental to the description of the binary: if we change any intrinsic
parameters we must recompute the orbital dynamics of the binary (typically through the relatively expensive process
of numerically integrating ordinary differential equations). Extrinsic parameters simply describe how the
binary is oriented in space and time relative to the detector; changing extrinsic parameters involves a 
relatively inexpensive rotation, translation or rescaling transformation. As we will show in the next subsection, 
for the non-spinning case considered here the intrinsic parameters are
\begin{equation} \label{eq:intrinsic}
\itrprm=\{\mc,\eta\}\ ,
\end{equation}
where $\mc = (m_1m_2)^{3/5}/(m_1+m_2)^{1/5}$) is the chirp mass and 
$\eta = m_1m_2/(m_1+m_2)^2$ is the symmetric mass ratio.
The extrinsic parameters are
\begin{equation} \label{eq:extrinsic}
\etrprm=\{t_{\rm geo},\alpha,\delta,\iota,D,\psi,\phi_c\}\ ,
\end{equation}
where $t_{\rm geo}$ is the time at which the waveform coalescence arrives at the Earth geocenter,
$\alpha$ and $\delta$ are the right ascension and declination, 
$\iota$ is the inclination angle of the binary's angular momentum vector and the line of sight to Earth, 
$D$ is the luminosity distance to the binary\footnote{For the sources considered in this paper, 
	the redshift correction is assumed to be negligible}, 
$\psi$ is the polarization angle, and $\phi_c$ is the orbital phase of the binary at coalescence.

Note that throughout the rest of the paper we will denote intrinsic parameters with $\itrprm$ 
and extrinsic parameters with $\etrprm$.

\subsection{Waveform decomposition}

The gravitational wave strain measured by the $k^{\rm th}$ interferometric detector in a network is given by
\begin{equation} \label{eq:measured_strain}
h_k(t) = F_{+,k}(\delta, \alpha, \psi) h_{+,k}(t) 
+ F_{\times,k}(\delta, \alpha, \psi) h_{\times,k}(t)\ ,
\end{equation}
where $F_{+,k}$, $F_{\times,k}$ are the antenna patterns of the detector and $h_{+,k}$, $h_{\times,k}$
are the two components of the gravitational wave strain, evaluated at the $k^{\rm th}$ detector. The antenna patterns depend only on the extrinsic sky location and polarization
angle, while the polarizations depend on both intrinsic and extrinsic parameters.   In turn, the polarizations can be
evaluated via the real and imaginary parts of a complex-valued plane wave $h(t|\mc,\eta,\iota,\phi_c)$:
\begin{align}
h_{+,k}(t)-i h_{\times, k}(t) &= h(t-t_k|\mc,\eta,\iota,\phi_c)
\end{align}

In this expression, $t_k$  denotes the time of arrival of the coalescence  at the $k^{th}$ detector,

\begin{equation} \label{eq:t_k}
t_k = t_{\rm geo} - \frac{\vec{x}_k \cdot \hat{N}(\alpha,\delta)}{c}\ .
\end{equation}

If $\vec{x}_k$ is a vector pointing from the geocenter to the $k^{th}$ detector and $\hat{N}(\alpha,\delta)$ is the direction of GW propagation, so each member of the network will have an offset relative to the geocenter time depending on sky location. 

At this stage, we make no assumptions about the functional form of $h(t|\mc,\eta,\iota,\phi_c)$
%
We can use a $-2$ spin-weighted spherical harmonic mode decomposition to further separate intrinsic and extrinsic
parameters appearing in the polarizations. 
In particular, we relate the polarizations to a set of $-2$ spin-weighted 
spherical harmonics modes with the equation
\begin{equation} \label{eq:h:Expansion}
h_{+,k}(t) - i\,h_{\times,k}(t) = \frac{D_{\rm ref}}{D} \sum_{lm} \hat{h}_{lm}(\mc,\eta,t_k;t)  \Y{-2}_{lm}(\iota,-\phi_c) \ ,
\end{equation}
For the purposes of this work, investigating radiation from inspiralling binaries, we will use $\hat{h}_{lm}$ as the -2
spin-weighted spherical harmonics modes provided by post-Newtonian calculations for adiabatic quasicircular inpiral \cite{gw-astro-mergers-approximations-SpinningPNHigherHarmonics},
evaluated at some fixed distance $D_{\rm ref}$ (in this work we choose $D_{\rm ref} = 100$ Mpc).
%
As a concrete example, at leading order for inspiral-only waveforms the polarizations are
\begin{widetext}
\begin{eqnarray} \label{eq:polarizations}
h_{+,k}(t) &=& - \frac{2 M \eta}{D}\,v^2(\mc,\eta,t_k;t)\,\left[ \left( 1 + \cos^2 \iota \right)\, \label{eq:h+}
	\cos 2 \left( \Phi(\mc,\eta,t_k;t) - \phi_c \right) + {\cal O}(v^3)\right]\ ,\\
h_{\times,k}(t) &=& - \frac{2 M \eta}{D}\,v^2(\mc,\eta,t_k;t)\,\left[ \left( 2 \cos \iota\right)\, \label{eq:hx}
	\sin 2 \left( \Phi(\mc,\eta,t_k;t) - \phi_c \right) + {\cal O}(v^3)\right]\ .
\end{eqnarray}
\end{widetext}
$\Phi(t)$ and $v(t)$ are the key orbital dynamical quantities, typically obtained via the energy balance equation,
which are expensive to compute.
%
If we define a complex-valued antenna pattern for each detector as
\begin{equation} \label{eq:complexF}
F_k = F_{+,k} + i \, F_{\times,k} \ ,
\end{equation}
then we can re-express the measured strain in the $k^{th}$ detector as
\begin{widetext}
\begin{equation} \label{eq:decomposed_strain}
h_k(\itrprm,\etrprm;t) = {\rm Re}\ \frac{D_{\rm ref}}{D}\, F_k(\alpha,\delta,\psi) \, \sum_{lm} 
\hat{h}_{lm}(\mc,\eta,t_k;t)  \Y{-2}_{lm}(\iota,-\phi_c)\ .
\end{equation}
\end{widetext}
Aside from $t_k$, we have now completely separated the intrinsic parameters (which enter only the $\hat{h}_{lm}$)
from the extrinsic parameters (which enter only the $F_k$ and $\Y{-2}_{lm}$).

Given a time-domain representation of the gravitational wave strain, we can define a 
frequency-domain version of this strain via a Fourier transform
\begin{equation} \label{eq:Fourier}
\tilde{h}(f) = \int_{-\infty}^\infty h(t) e^{- 2 \pi i f t} dt\ .
\end{equation}
Time translation of frequency-domain waveforms is trivial.
If $\tilde{h}(t_k;f)$ is the Fourier-domain representation of a strain for some arrival time $t_k$, then
the same strain arriving at another time $t_k'$ can be simply related by
\begin{equation} \label{eq:time_shift}
\tilde{h}(t_k';f) = \tilde{h}(t_k;f) \, e^{- 2 \pi i f (t_k' - t_k)}\ .
\end{equation}
Thus, if we work with frequency-domain waveforms, the arrival time $t_k$ 
can be factored out as $\exp( - 2 \pi i f t_k)$
and we can complete the separation of intrinsic and extrinsic parameters.

For the explicit decomposition above, we have focused on non-spinning waveforms 
and found that they can be separated into two intrinsic parameters and seven
extrinsic parameters. In the general case, which could include precession,
tidal effects and any other physics, this intrinsic and extrinsic separation is still
possible. In fact, there will always be seven extrinsic parameters which enter only
through inexpensive geometric factors, while any additional parameters will always
be encoded in the expensive $h_{lm}$ modes.

To see that this is true, first note that Eq.~(\ref{eq:time_shift}) holds for an arbitrary strain
and so can always be used to factor out the dependence on time of arrival. Similarly,
a gravitational wave far from its source will always fall off as $1/D$. Furthermore, 
the antenna patterns $F_+$ and $F_\times$ depend on the detector geometry,
not the source, and so are unchanged. This gives a total of five extrinsic parameters
that in general can be factored out exactly as in our non-spinning waveform. Lastly, 
we have two angles which enter into the $\Y{-2}_{lm}$. The physical interpretation 
of these angles depends on how the frame in which the harmonic mode decomposition is 
performed is defined. One should choose a frame which is convenient for expressing 
and computing the $h_{lm}$. For the non-spinning case, this means aligning the frame 
with the orbital angular momentum, $\hat{L}$, in which case one can show that the 
zenith angle is $\iota$ and the azimuth is $- \phi_c$. In the precessing case, one would
most likely use a frame aligned with the total angular momentum, $\hat{J}$, e.g. using the
parameterization described in~\cite{Farr:2014qka}. In the notation of that paper, the extrinsic
spherical harmonic dependence is $\Y{-2}_{lm}(\theta_{JN},-\phi_{JL})$, 
where $\theta_{JN}$ is the inclination of the \emph{total} angular momentum to line of sight,
and $\phi_{JL}$ marks the azimuthal position of $\hat{L}$ on its precession cone about $\hat{J}$.
Note that if we use such a frame $\iota$ and $\phi_c$ must be encoded in the $h_{lm}$, 
as must any other parameters such as spin vectors or tidal deformabilities.


