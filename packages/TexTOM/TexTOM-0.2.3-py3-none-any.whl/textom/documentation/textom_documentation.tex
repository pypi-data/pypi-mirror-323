\documentclass{article}
\usepackage[utf8]{inputenc}
\usepackage{hyperref}
\usepackage{listings}
\usepackage{xcolor} % For custom colors
\usepackage{sectsty} % For customizing section fonts
% \subsectionfont{\raggedright} % Allow line breaks and ragged right alignment             
\usepackage{geometry}
% \geometry{margin=1in}
\usepackage{graphicx}
                
\title{TexTOM - a software for texture simulations}
\author{Moritz Frewein}
\begin{document}
\maketitle
\label{toc}
\tableofcontents

% Define a custom style for docstrings
\lstdefinelanguage{docstring}{
    basicstyle=\ttfamily\small, % Monospaced font
    backgroundcolor=\color[HTML]{F5F5F5}, % Light gray background
    frame=single, % Border around the docstring
    rulecolor=\color[HTML]{D6D6D6}, % Border color
    keywordstyle=\color{blue}, % Optional: color for keywords
    breaklines=true, % Wrap long lines
}
                
\subsectionfont{\large\ttfamily\raggedright}

\label{sec:introduction}

Ground based gravitational wave detector networks (notably LIGO \cite{gw-detectors-LIGO-original-preferred} and Virgo
\cite{gw-detectors-VIRGO-original-preferred})  are sensitive to the gravitational wave signal from coalescing compact
binaries, both the relatively well understood signal from  the lowest-mass compact binaries
$M=m_1+m_2\le 16 M_\odot$
\cite{2003PhRvD..67j4025B,2004PhRvD..70j4003B,2004PhRvD..70f4028D,BCV:PTF,2005PhRvD..71b4039K,2005PhRvD..72h4027B,2006PhRvD..73l4012K,2007MNRAS.374..721T,2008PhRvD..78j4007H,gr-astro-eccentric-NR-2008,gw-astro-mergers-approximations-SpinningPNHigherHarmonics,gw-astro-PN-Comparison-AlessandraSathya2009}
and the less-well-understood strong-field epoch
\cite{2011PhRvD..83l2005A,2009CQGra..26p5008A, 2014PhRvD..89d2002K,2009PhRvD..79l4028B,2010PhRvD..82f4016S,2011CQGra..28m4002M,2013PhRvD..87b4009M,2013CQGra..31b5012H}.    
%
% POINT: Inference via Bayesian methods
Interpreting gravitational wave data requires systematically comparing all possible candidate signals to the data,
constructing a Bayesian posterior probability distribution for candidate binary parameters \citeMCMC{}.   
%
Owing to the complexity and multimodality of the model space, historically successful strategies have adopted  generic, serial
algorithms for parameter estimation, such as variants of Markov Chain Monte Carlo or nested sampling
\cite{2011RvMP...83..943V,gw-astro-PE-lalinference-v1}.   
%
In physics, similar path-based methods have been enormously successful at a broad range of physical problems, by
exploring all  possible paths  through a configuration space; see,
e.g.,  \cite{2001RvMP...73...33F,1987PhLB..195..216D}.  
% POINT: Why do something new?  
%   -Because they are slow, being serial
Though successful, these generic algorithms are functionally serial, requiring intensive communication to coordinate the
current state, and are therefore well-known to scale poorly to large numbers of processors.  
%
These algorithms' convergence is also limited by \emph{ergodicity}.  
 No theorem  guarantees these algorithm \emph{must}  explore the entire model space, let alone efficiently; no expression can robustly assess convergence, using available
 sampled data.  
%
By contrast, well-understood, efficient, and highly-parallelizable Monte Carlo integration strategies are frequently applied to problems with dimensions comparable or
higher than coalescing compact binaries; see, e.g.,  \cite{lepage1980vegas,1980PhRvD..21.2308C,book-math-Jaeckel-MonteCarlo,mm-QuasiMonteCarlo-Papageorgiou2001}.    In this work, we apply such methods to gravitational wave
parameter estimation for the first time.  
%
% http://en.wikipedia.org/wiki/Quasi-Monte_Carlo_methods_in_finance
%
%  Beware re quantum monte carlo -- often it is basically an MCMC
%    - http://adsabs.harvard.edu/abs/2001RvMP...73...33F
%    - 1987PhLB..195..216D,


% POINT :
%  - because they are slow, not communicating 
Gravitational wave parameter estimation has also been limited by the computational cost associated with comparing the
data with a candidate waveform.  This cost has two factors: first, the cost of waveform generation, as discussed in
\cite{gwastro-mergers-PE-ReducedOrder-2013,2013PhRvD..87l2002S,2013PhRvD..87d4008C,gwastro-mergers-IMRPhenomP,gwastro-SpinTaylorF2-2013} and references therein; and, second,
the cost of repeatedly operating on the long time- or frequency-series itself, once per comparison (e.g., fourier
transforms).   
%
Recently, several methods have been proposed to perform this comparison more efficiently
\cite{gwastro-mergers-PE-ReducedOrder-2013,2013PhRvD..87l2002S,2013PhRvD..87d4008C,gw-astro-ReducedOrderQuadraturePE-TiglioEtAl2014}, by interpolating some combination
of the the waveform or likelihood or by adopting a sparse representation to reduce the computational cost of data
handling.  
%
In this work, we introduce a new, simple, and extremely robust scheme to reduce the cost per comparison.  
%
To our knowledge, this work is the first time any such method has been implemented at production scale, particularly for
the most accurate and computationally-expensive waveform models like EOB
\cite{gw-astro-EOBspin-Tarrachini2012,gw-astro-EOBNR-Calibrated-2009}.  


% POINT: Blindly redo the anaysis, versus using information from search?
Historically, parameter estimation strategies very practically make almost no use of the information reported by
existing search codes.  This careful approach ensured, for example, that inferences from the data would never be used as priors on a
reanalysis of the same data.  
%
That said, experience suggests the search codes provide a reasonable first approximation, 
particularly for the most tightly constrained parameters, relative to their priors (i.e., event time; chirp time; sky location).  
%
Some parameter estimation codes have used this information to improve their performance, more efficiently searching the
sky \cite{2013APS..APRG10003F,2013arXiv1309.7709F}. % \editremark{check with Ben F}.  
That said, historical experience with serial codes suggest \emph{sampling} the maximum (not finding it) is what limits
performance; based on that experience, one might expect minimal additional asymptotic speed improvement when provided
with a good initial guess.
%
By contrast, Monte Carlo integration trivially provides robust sampling of any weakly-constrained parameters (e.g.,
distance, inclination, polarization).    By using search results to sample near expected results, Monte Carlo methods
can be both generic and extremely efficient.   We therefore describe a procedure systematically and organically incorporates information provided
by the search to accelerate our specific algorithm's performance.  
%
Combining these three factors (parallelizable monte carlo integration; efficient likelihoods; and search information),
we can provide reliable parameter estimates for merging double neutron star binaries within one hour on existing
hardware for instruments available in the next two years.


% POINT: Low latency as ratioale
In the era of multimessenger astronomy, rapid and robust inference about candidate compact binary gravitational wave
events will be a critical science product for LIGO, as  colleagues with other instruments  perform followup and
coincident observations \cite{LIGO-2013-WhitePaper-CoordinatedEMObserving}.  
%
The most tantalizing proposed electromagnetic counterparts are expected to be brief, potentially disappearing within
days if not much
sooner \cite{2012ApJ...746...48M,2014MNRAS.439..757G,2014MNRAS.437L...6K,2014MNRAS.437.1821M,2014ApJ...780...31T,2013ApJ...775...18B}.  
%
Given limited resources, reliable low-latency parameter estimation of gravitational wave signals will significantly enhance the science output of
multimessener, time-domain astronomy.  

% ``burnin is short; finding the best fit is fast;
%what's expensive is mapping out the likelihood near the maximum'' \editremark{fixme: can't use in publication}.
% 


% POINT: Outline
This work is organized as follows.
In Section \ref{sec:Executive} we provide an executive summary, providing the principles that enable our algorithm to
provide rapid, accurate results for the test cases explored here:  time-domain models for nonprecessing binaries.  
%
Then, in Section \ref{sec:Methods}, we describe our algorithm in detail.  
%
To demonstrate  our algorithm provides high performance in a production environment,  Section \ref{sec:Results} we
provide concrete results drawn from a large sample of events from the ``2015 double neutron star mock data challenge'',
results from which will be described in \cite{first2years}.  
%
In Section \ref{sec:Discussion} we reflect on the broader significance of our result, in the context of other parameter
estimation work inside and outside the LSC.  
%
We conclude with Section \ref{sec:Conclude}.  

\section{Functions}\label{sec:functions}
\subsection*{\texttt{set\_path(path)}}
\addcontentsline{toc}{subsection}{set\_path}

\begin{lstlisting}[language=docstring]
Set the path where integrated data and analysis is stored

Parameters
----------
path : str
    full path to the directory, must contain a folder '/data_integrated'
\end{lstlisting}

\begin{flushright}

\hyperref[toc]{ToC}

\end{flushright}

\input{functions/set_path}

\vspace{5mm}

\hrule

\subsection*{\texttt{check\_state()}}
\addcontentsline{toc}{subsection}{check\_state}

\begin{lstlisting}[language=docstring]
Prints in terminal which parts of the reconstruction are ready
    
\end{lstlisting}

\begin{flushright}

\hyperref[toc]{ToC}

\end{flushright}

\input{functions/check_state}

\vspace{5mm}

\hrule

\subsection*{\texttt{integrate()}}
\addcontentsline{toc}{subsection}{integrate}

\begin{lstlisting}[language=docstring]
Integrates raw 2D diffraction data via pyFAI

All necessary input will be handled via the file integration_parameters.py
\end{lstlisting}

\begin{flushright}

\hyperref[toc]{ToC}

\end{flushright}

\input{functions/integrate}

\vspace{5mm}

\hrule

\subsection*{\texttt{align\_data(pattern='.h5', sub\_data='data\_integrated', q\_index\_range=(0, 5), q\_range=False, crop\_image=False, mode='optical\_flow', redo\_import=False, flip\_fov=False, regroup\_max=16, align\_horizontal=True, align\_vertical=True, pre\_rec\_it=5, pre\_max\_it=5, last\_rec\_it=40, last\_max\_it=5)}}
\addcontentsline{toc}{subsection}{align\_data}

\begin{lstlisting}[language=docstring]
Align data using the Mumott optical flow alignment

Requires that data has been integrated and that sample_dir contains
a subfolder with data

Parameters
----------
pattern : str, optional
    substring contained in all files you want to use, by default '.h5'
sub_data : str, optional
    subfolder containing the data, by default 'data_integrated'
q_index_range : tuple, optional
    determines which q-values are used for alignment (sums over them), by default (0,5)
q_range : tuple, optional
    give the q-range in nm instead of indices e.g. (15.8,18.1), by default False
crop_image : bool or tuple of int, optional
    give the range you want to use in x and y, e.g. ((0,-1),(10,-10)), by default False
mode : str, optional
    choose alignment mode, 'optical_flow' or 'phase_matching', by default 'optical_flow'
redo_import : bool, optional
    set True if you want to recalculate data_mumott.h5, by default False
flip_fov : bool, optional
    only to be used if the fov is in the wrong order in the integrated
    data files, by default False
regroup_max : int, optional
    maximum size of groups when downsampling for faster processing, by default 16
align_horizontal : bool, optional
    align your data horizontally, by default True
align_vertical : bool, optional
    align your data vertically, by default True
pre_rec_it : int, optional
    reconstruciton iterations for downsampled data, by default 5
pre_max_it : int, optional
    alignment iterations for downsampled data, by default 5
last_rec_it : int, optional
    reconstruciton iterations for full data, by default 40
last_max_it : int, optional
    alignment iterations for full data, by default 5
\end{lstlisting}

\begin{flushright}

\hyperref[toc]{ToC}

\end{flushright}

\input{functions/align_data}

\vspace{5mm}

\hrule

\subsection*{\texttt{check\_alignment\_consistency()}}
\addcontentsline{toc}{subsection}{check\_alignment\_consistency}

\begin{lstlisting}[language=docstring]
Plots the squared residuals between data and the projected tomograms

    
\end{lstlisting}

\begin{flushright}

\hyperref[toc]{ToC}

\end{flushright}

\input{functions/check_alignment_consistency}

\vspace{5mm}

\hrule

\subsection*{\texttt{check\_alignment\_projection(g=0)}}
\addcontentsline{toc}{subsection}{check\_alignment\_projection}

\begin{lstlisting}[language=docstring]
Plots the data and the projected tomogram of projection g

Parameters
----------
g : int, optional
    projection running index, by default 0
\end{lstlisting}

\begin{flushright}

\hyperref[toc]{ToC}

\end{flushright}

\input{functions/check_alignment_projection}

\vspace{5mm}

\hrule

\subsection*{\texttt{make\_model()}}
\addcontentsline{toc}{subsection}{make\_model}

\begin{lstlisting}[language=docstring]
Calculates the TexTOM model for reconstructions

Is automatically performed by the functions that require it
\end{lstlisting}

\begin{flushright}

\hyperref[toc]{ToC}

\end{flushright}

\input{functions/make_model}

\vspace{5mm}

\hrule

\subsection*{\texttt{preprocess\_data(pattern='.h5', flip\_fov=False, baselines=1, use\_ion=True)}}
\addcontentsline{toc}{subsection}{preprocess\_data}

\begin{lstlisting}[language=docstring]
Loads integrated data and pre-processes them for TexTOM

Parameters
----------
pattern : str, optional
    substring contained in all files you want to use, by default '.h5'
flip_fov : bool, optional
    only to be used if the fov is in the wrong order in the integrated
    data files, by default False
baselines : bool, optional
    choose if you want to draw polynomial baselines
    set the polynomial order in the argument or False, by default 1
use_ion : bool, optional
    choose if you want to normalize data by the field 'ion' in the 
    data files, by default True
\end{lstlisting}

\begin{flushright}

\hyperref[toc]{ToC}

\end{flushright}

\input{functions/preprocess_data}

\vspace{5mm}

\hrule

\subsection*{\texttt{make\_fit(redo=True)}}
\addcontentsline{toc}{subsection}{make\_fit}

\begin{lstlisting}[language=docstring]
Initializes a TexTOM fit object for reconstructions

Is automatically performed by the functions that require it

Parameters
----------
redo : bool, optional
    set True for recalculating, by default True
\end{lstlisting}

\begin{flushright}

\hyperref[toc]{ToC}

\end{flushright}

\input{functions/make_fit}

\vspace{5mm}

\hrule

\subsection*{\texttt{optimize(order=0, mode=0, proj='full', zero\_peak=None, redo\_fit=False, tol=0.001, minstep=1e-09, itermax=3000, alg='quadratic', save\_h5=True)}}
\addcontentsline{toc}{subsection}{optimize}

\begin{lstlisting}[language=docstring]
Performs a single TexTOM parameter optimization

Parameters
----------
order : int, optional
    maximum sHSH order to be used, by default 0
mode : int, optional
    set 0 for only optimizing order 0, 1 for highest order, 2 for all,
    by default 0
proj : str, optional
    choose projections to be optimized: 'full', 'half', 'third', 'notilt', 
    by default 'full'
zero_peak : int or None
    index of the peak you want to use for 0-order fitting (should be as
    isotropic as possible), if None uses the whole dataset, default None
redo_fit : bool, optional
    recalculate the fit object, by default False
tol : float, optional
    tolerance for precision break criterion, by default 1e-3
minstep : float, optional
    minimum stepsize in line search, by default 1e-9
itermax : int, optional
    maximum number of iterations, by default 3000
alg : str, optional
    choose algorithm between 'backtracking', 'simple', 'quadratic', 
    by default 'quadratic'
save_h5 : bool, optional
    choose if you want to save the result to the directory analysis/fits, 
    by default True    
\end{lstlisting}

\begin{flushright}

\hyperref[toc]{ToC}

\end{flushright}

\input{functions/optimize}

\vspace{5mm}

\hrule

\subsection*{\texttt{optimize\_auto(max\_order=8, start\_order=None, zero\_peak=None, tol\_0=1e-07, tol\_1=0.001, tol\_2=0.0001, minstep\_0=1e-09, minstep\_1=1e-09, minstep\_2=1e-09, projections='full', alg='quadratic', adj\_scal=False, redo\_fit=False)}}
\addcontentsline{toc}{subsection}{optimize\_auto}

\begin{lstlisting}[language=docstring]
Automated TexTOM reconstruction workflow

Parameters
----------
max_order : int, optional
    maximum HSH order to be used, by default 8    
start_order : int or None, optional
    lowest order to be fitted, if None continues where you are standing, 
    by default None
zero_peak : int or None
    index of the peak you want to use for 0-order fitting (should be as
    isotropic as possible), if None uses the whole dataset, default None
redo_fit : bool, optional
    recalculate the fit object, by default False
proj : str, optional
    choose projections to be optimized: 'full', 'half', 'third', 'notilt', by default 'full'
alg : str, optional
    choose algorithm between 'backtracking', 'simple', 'quadratic', 
    by default 'quadratic'
\end{lstlisting}

\begin{flushright}

\hyperref[toc]{ToC}

\end{flushright}

\input{functions/optimize_auto}

\vspace{5mm}

\hrule

\subsection*{\texttt{adjust\_data\_scaling()}}
\addcontentsline{toc}{subsection}{adjust\_data\_scaling}

\begin{lstlisting}[language=docstring]
Reestimates the data based on the assumption that normalization constants
contain noise. To be used after fitting the 0th order
\end{lstlisting}

\begin{flushright}

\hyperref[toc]{ToC}

\end{flushright}

\input{functions/adjust_data_scaling}

\vspace{5mm}

\hrule

\subsection*{\texttt{list\_opt()}}
\addcontentsline{toc}{subsection}{list\_opt}

\begin{lstlisting}[language=docstring]
Shows all stored optimizations
    
\end{lstlisting}

\begin{flushright}

\hyperref[toc]{ToC}

\end{flushright}

\input{functions/list_opt}

\vspace{5mm}

\hrule

\subsection*{\texttt{load\_opt(h5path='last')}}
\addcontentsline{toc}{subsection}{load\_opt}

\begin{lstlisting}[language=docstring]
Loads a previous Textom optimization into memory
seful: load_opt(results['optimization'])

Parameters
----------
h5path : str, optional
    filepath, just filename or full path
    if 'last', uses the youngest file is used in analysis/fits/, 
    by default 'last'
\end{lstlisting}

\begin{flushright}

\hyperref[toc]{ToC}

\end{flushright}

\input{functions/load_opt}

\vspace{5mm}

\hrule

\subsection*{\texttt{check\_lossfunction()}}
\addcontentsline{toc}{subsection}{check\_lossfunction}

\begin{lstlisting}[language=docstring]
No docstring available.
\end{lstlisting}

\begin{flushright}

\hyperref[toc]{ToC}

\end{flushright}

\input{functions/check_lossfunction}

\vspace{5mm}

\hrule

\subsection*{\texttt{check\_fit\_average()}}
\addcontentsline{toc}{subsection}{check\_fit\_average}

\begin{lstlisting}[language=docstring]
Plots the reconstructed average intensity for each projection with data

Parameters
----------
\end{lstlisting}

\begin{flushright}

\hyperref[toc]{ToC}

\end{flushright}

\input{functions/check_fit_average}

\vspace{5mm}

\hrule

\subsection*{\texttt{check\_fit\_random(N=10, mode='line')}}
\addcontentsline{toc}{subsection}{check\_fit\_random}

\begin{lstlisting}[language=docstring]
Generates TexTOM reconstructions and plots them with data for random points

Parameters
----------
N : int, optional
    Number of images created, by default 10    
mode : str, optional
    plotting mode, 'line' or 'color', by default line
\end{lstlisting}

\begin{flushright}

\hyperref[toc]{ToC}

\end{flushright}

\input{functions/check_fit_random}

\vspace{5mm}

\hrule

\subsection*{\texttt{check\_residuals()}}
\addcontentsline{toc}{subsection}{check\_residuals}

\begin{lstlisting}[language=docstring]
Plots the squared residuals summed over each projection
    
\end{lstlisting}

\begin{flushright}

\hyperref[toc]{ToC}

\end{flushright}

\input{functions/check_residuals}

\vspace{5mm}

\hrule

\subsection*{\texttt{check\_projections\_average(G=None)}}
\addcontentsline{toc}{subsection}{check\_projections\_average}

\begin{lstlisting}[language=docstring]
Plots the reconstructed average intensity for chosen projections with data

Parameters
----------
G : int or ndarray or None, optional
    projection indices, if None takes 10 equidistant ones, by default None
\end{lstlisting}

\begin{flushright}

\hyperref[toc]{ToC}

\end{flushright}

\input{functions/check_projections_average}

\vspace{5mm}

\hrule

\subsection*{\texttt{check\_projections\_residuals(G=None)}}
\addcontentsline{toc}{subsection}{check\_projections\_residuals}

\begin{lstlisting}[language=docstring]
Plots the residuals per pix3l for chosen projections with data

Parameters
----------
G : int or ndarray or None, optional
    projection indices, if None takes 10 equidistant ones, by default None
\end{lstlisting}

\begin{flushright}

\hyperref[toc]{ToC}

\end{flushright}

\input{functions/check_projections_residuals}

\vspace{5mm}

\hrule

\subsection*{\texttt{check\_projections\_orientations(G=None)}}
\addcontentsline{toc}{subsection}{check\_projections\_orientations}

\begin{lstlisting}[language=docstring]
Plots the reconstructed average orientations for chosen projections with data

Parameters
----------
G : int or ndarray or None, optional
    projection indices, if None takes 10 equidistant ones, by default None
\end{lstlisting}

\begin{flushright}

\hyperref[toc]{ToC}

\end{flushright}

\input{functions/check_projections_orientations}

\vspace{5mm}

\hrule

\subsection*{\texttt{calculate\_orientation\_statistics()}}
\addcontentsline{toc}{subsection}{calculate\_orientation\_statistics}

\begin{lstlisting}[language=docstring]
Calculates prefered orientations and stds and saves them to results dict

    
\end{lstlisting}

\begin{flushright}

\hyperref[toc]{ToC}

\end{flushright}

\input{functions/calculate_orientation_statistics}

\vspace{5mm}

\hrule

\subsection*{\texttt{calculate\_segments(thresh=10, min\_segment\_size=30, max\_segments\_number=31)}}
\addcontentsline{toc}{subsection}{calculate\_segments}

\begin{lstlisting}[language=docstring]
Segments the sample based on misorientation borders

Parameters
----------
thresh : float, optional
    misorientation angle threshold inside segment in degree, by default 10
min_segment_size : int, optional
    minimum number of voxels in segment, by default 30
max_segments_number : int, optional
    maximum number of segments (ordered by size), by default 32
\end{lstlisting}

\begin{flushright}

\hyperref[toc]{ToC}

\end{flushright}

\input{functions/calculate_segments}

\vspace{5mm}

\hrule

\subsection*{\texttt{show\_volume(data='scaling', plane='z', colormap='inferno', cut=1, save=False, show=True)}}
\addcontentsline{toc}{subsection}{show\_volume}

\begin{lstlisting}[language=docstring]
Visualizes the whole sample by slices, colored by a value of your choice

Parameters
----------
data : str or list, optional
    name of one entry in the results dict or list of entries, 
    by default 'scaling'
plane : str, optional
    sliceplane 'x'/'y'/'z', by default 'z'
colormap : str, optional
    identifier of matplotlib colormap, default 'inferno'
    https://matplotlib.org/stable/users/explain/colors/colormaps.html
cut : int, optional
    cut colorscale at upper and lower percentile, by default 0.1
save : bool, optional
    saves tomogram as .gif to results/images/, by default False
show : bool, optional
    open the figure upon calling the function, by default True
\end{lstlisting}

\begin{flushright}

\hyperref[toc]{ToC}

\end{flushright}

\input{functions/show_volume}

\vspace{5mm}

\hrule

\subsection*{\texttt{show\_slice\_ipf(h, plane='z')}}
\addcontentsline{toc}{subsection}{show\_slice\_ipf}

\begin{lstlisting}[language=docstring]
Plots an inverse pole figure of a sample slice

Parameters
----------
h : int
    height of the slice
plane : str, optional
    slice direction: x/y/z, by default 'z'
\end{lstlisting}

\begin{flushright}

\hyperref[toc]{ToC}

\end{flushright}

\input{functions/show_slice_ipf}

\vspace{5mm}

\hrule

\subsection*{\texttt{show\_volume\_ipf(plane='z', save=False, show=True)}}
\addcontentsline{toc}{subsection}{show\_volume\_ipf}

\begin{lstlisting}[language=docstring]
Plots inverse pole figures as a tomogram with a slider to scroll through the sample

Parameters
----------
plane : str, optional
    slice direction: x/y/z, by default 'z'
save : bool, optional
    saves tomogram as .gif to results/images/, by default False
show : bool, optional
    open the figure upon calling the function, by default True
\end{lstlisting}

\begin{flushright}

\hyperref[toc]{ToC}

\end{flushright}

\input{functions/show_volume_ipf}

\vspace{5mm}

\hrule

\subsection*{\texttt{show\_histogram(x, nbins=50, cut=0.1, segments=None, save=False)}}
\addcontentsline{toc}{subsection}{show\_histogram}

\begin{lstlisting}[language=docstring]
plots a histogram of a result parameter

Parameters
----------
x : str,
    name of a scalar from results
bins : int, optional
    number of bins, by default 50
cut : int, optional
    cut upper and lower percentile, by default 0.1
segments : list of int, optional
    list of segments or None for all data, by default None
save : bool/str, optional
    saves image with specified file extension, e.g. 'png', 'pdf'
    if True uses png, by default False
\end{lstlisting}

\begin{flushright}

\hyperref[toc]{ToC}

\end{flushright}

\input{functions/show_histogram}

\vspace{5mm}

\hrule

\subsection*{\texttt{show\_correlations(x, y, nbins=50, cut=(0.1, 0.1), segments=None, save=False)}}
\addcontentsline{toc}{subsection}{show\_correlations}

\begin{lstlisting}[language=docstring]
Plots a 2D histogram between 2 result parameters

Parameters
----------
x : str,
    name of a scalar from results
y : str,
    name of a scalar from results
bins : int, optional
    number of bins, by default 50
cut : tuple, optional
    cut upper and lower percentile of both parameters, by default (0.1,0.1)
segments : list, optional
    list of segments or None for all data, by default None
save : bool/str, optional
    saves image with specified file extension, e.g. 'png', 'pdf'
    if True uses png, by default False
\end{lstlisting}

\begin{flushright}

\hyperref[toc]{ToC}

\end{flushright}

\input{functions/show_correlations}

\vspace{5mm}

\hrule

\subsection*{\texttt{show\_voxel\_odf(x, y, z, num\_samples=1000)}}
\addcontentsline{toc}{subsection}{show\_voxel\_odf}

\begin{lstlisting}[language=docstring]
Show a 3D plot of the ODF in the chosen voxel

Parameters
----------
x : int
    voxel x-coordinate
y : int
    voxel y-coordinate
z : int
    voxel z-coordinate
num_samples : int/float, optional
    number of samples for plot generation, by default 1000
\end{lstlisting}

\begin{flushright}

\hyperref[toc]{ToC}

\end{flushright}

\input{functions/show_voxel_odf}

\vspace{5mm}

\hrule

\subsection*{\texttt{show\_voxel\_polefigure(x, y, z, hkl=(1, 0, 0), mode='density', alpha=0.1, num\_samples=10000.0)}}
\addcontentsline{toc}{subsection}{show\_voxel\_polefigure}

\begin{lstlisting}[language=docstring]
Show a polefigure plot for the chosen voxel and hkl

Parameters
----------
x : int
    voxel x-coordinate
y : int
    voxel y-coordinate
z : int
    voxel z-coordinate
hkl : tuple, optional
    Miller indices, by default (1,0,0)
mode : str, optional
    plotting style 'scatter' or 'density', by default 'density'
alpha : float, optional
    opacity of points, only for scatter, by default 0.1
num_samples : int/float, optional
    number of samples for plot generation, by default 1e4
\end{lstlisting}

\begin{flushright}

\hyperref[toc]{ToC}

\end{flushright}

\input{functions/show_voxel_polefigure}

\vspace{5mm}

\hrule

\subsection*{\texttt{save\_results()}}
\addcontentsline{toc}{subsection}{save\_results}

\begin{lstlisting}[language=docstring]
Saves the results dictionary to a h5 file in the results/ directory

    
\end{lstlisting}

\begin{flushright}

\hyperref[toc]{ToC}

\end{flushright}

\input{functions/save_results}

\vspace{5mm}

\hrule

\subsection*{\texttt{list\_results()}}
\addcontentsline{toc}{subsection}{list\_results}

\begin{lstlisting}[language=docstring]
Shows all results .h5 files in results directory
    
\end{lstlisting}

\begin{flushright}

\hyperref[toc]{ToC}

\end{flushright}

\input{functions/list_results}

\vspace{5mm}

\hrule

\subsection*{\texttt{load\_results(h5path='last', make\_bg\_nan=False)}}
\addcontentsline{toc}{subsection}{load\_results}

\begin{lstlisting}[language=docstring]
Loads the results from a h5 file do the results dictionary

Parameters
----------
h5path : str, optional
    filepath, just filename or full path
    if 'last', uses the youngest file is used in results/, 
    by default 'last'
make_bg_nan : bool, optional
    if true, replaces all excluded voxels by NaN
\end{lstlisting}

\begin{flushright}

\hyperref[toc]{ToC}

\end{flushright}

\input{functions/load_results}

\vspace{5mm}

\hrule

\subsection*{\texttt{list\_results\_loaded()}}
\addcontentsline{toc}{subsection}{list\_results\_loaded}

\begin{lstlisting}[language=docstring]
Shows all results currently in memory
    
\end{lstlisting}

\begin{flushright}

\hyperref[toc]{ToC}

\end{flushright}

\input{functions/list_results_loaded}

\vspace{5mm}

\hrule

\subsection*{\texttt{save\_images(x, ext='raw')}}
\addcontentsline{toc}{subsection}{save\_images}

\begin{lstlisting}[language=docstring]
Export results as .raw or .tiff files for dragonfly

Parameters
----------
x : str,
    name of a scalar from results, e.g. 'scaling'
ext : str,
    desired file type by extension, can do 'raw' or 'tiff', default: 'raw'
\end{lstlisting}

\begin{flushright}

\hyperref[toc]{ToC}

\end{flushright}

\input{functions/save_images}

\vspace{5mm}

\hrule

\subsection*{\texttt{help(method=None, module=None, filter='')}}
\addcontentsline{toc}{subsection}{help}

\begin{lstlisting}[language=docstring]
Prints information about functions in this library

Parameters
----------
method : str or None, optional
    get more information about a function or None for overview over all functions, by default None
module : str or None, optional
    choose python module or None for the base TexTOM library, by default None
filter : str, optional
    filter the displayed functions by a substring, by default ''
\end{lstlisting}

\begin{flushright}

\hyperref[toc]{ToC}

\end{flushright}

\input{functions/help}

\vspace{5mm}

\hrule

\end{document}\end{document}