\chapter {FAMILIES OF GENERATORS}

\def\eps  {$\epsilon\quad$}
\def\epsm  {--$\epsilon_1\quad$}

The tools described in this chapter are convenient for examining
systematically the interaction between specific tests
and certain families of RNGs. 
The framework is as follows.
For each family, an RNG of period length near $2^i$ has been selected,
on the basis on some theoretical criteria
that depend on the family, for all integers $i$ in some interval
(from 10 to 40, for example).
% Here, for several families of RNGs, 
The parameters of the pre-selected instances are stored in text files
in directory {\tt param}.

Typically, for a given RNG that fails a test, when the
sample size of the test is increased, the $p$-value would remain
``reasonable'' for a while, say  for $n$ up to some 
threshold $n_0$ (roughly), and will then converges to 0 or 1 exponentially 
fast as a function of $n$.
It is interesting to examine the relationship between $n_0$ and $i$.
The idea is to fit a crude regression model of $n_0$ as a function
of $i$.  For example, one may consider a model of the form
\eq
  \log_2 n_0 = \gamma i + \nu + \delta,
\endeq
where $\gamma$ and $\nu$ are constants and $\delta$ represents
the noise. The result may give an idea of what period length $\rho$
of the RNG is required, within a given family, to be safe with respect
to the test that is considered, for a given computer budget.

This methodology has been applied in 
\cite{rGRA01a,rLEC98h,rLEC00c,rLEC01a,rLEC02c,rLEC03c},
using an earlier version of the present library,
and gave surprisingly good results in many cases.
For full-period LCGs with good spectral test behavior, for example, 
the relationships $n_0 \approx 16\,\rho^{1/2}$
for the collision test and $n_0 \approx 16\,\rho^{1/3}$
for the birthday spacings test have been obtained.
This means that no LCG is safe with respect to these particular tests
unless its period length $\rho$ is so large that generating 
$\rho^{1/3}$ numbers is practically unfeasible.  
A period length of $2^{48}$ or less, for example, does not satisfy
this requirement.


%%%%%%%%%%%%%%%%%%%%%%
\paragraph*{Common parameters.} \


The first argument of each testing function in the {\tt f} modules 
is the family {\tt fam} of random number generators to be tested
(see module {\tt ffam} for details).
That family must be created by calling the  appropriate function in
the module {\tt fcong} or {\tt ffsr}, and deleted when no longer needed.

The second argument (the third one in the {\tt fmultin} module)
of each testing function is a structure {\tt res} that can keep the tables
 of $p$-values and other results (see module {\tt fres} for details).
This is useful if one wishes to do something else with the tables of
results after all tests are ended. If one does not
want to post-process or use the tables of results after a {\tt f} test,
it suffices to set the {\tt res} argument to the {\tt NULL} pointer.
Then, the structure is created and deleted automatically inside the 
testing function. In any case, the tables of results will be printed
automatically on standard output.

The third argument (the fourth one in the {\tt fmultin} module)
of each testing function is a structure 
{\tt cho} that allows the user to choose varying sample sizes and
other parameters of the tests as a function of the generators {\it lsize}
and the fixed parameters (see module {\tt fcho} for details).

For each of these three arguments (except possibly for {\tt res} as
explained above), one must call the appropriate {\tt Create} function
before using them, and call the corresponding {\tt Delete} function
when they are no longer needed.

The last four arguments of each testing function are the integers 
{\tt Nr, j1, j2} and {\tt jstep}. The test functions will be applied on
the first {\tt Nr} generators of the family {\tt fam}.
If there are less than {\tt Nr} generators in the family, then {\tt Nr}
 will be reset to the number of generators in the tested family.
 For each of the generator, tests will be
applied with varying sample sizes determined by the parameter $j$ for
$j$ varying from {\tt j1} to {\tt j2} by step of {\tt jstep}.



\begin {table}
\begin {center}
\caption {Some good LCGs according to the spectral test in up to
          dimension 8.}
\label {tab:lcg1}
\smallskip 
\begin {tabular}{|@{\quad}l@{\quad}|@{\quad}r@{\quad}|}
\hline
\qquad $m$             & $a$ \\
\hline
$2^{10}-3\phantom{0} = 1021$     &     \rule{0pt}{14pt}   65      \\
$2^{11}-9\phantom{0} = 2039$     & 995            \\
 $2^{12}-3\phantom{0} = 4093$    & 209            \\
 $2^{13}-1\phantom{0} = 8191$    & 884            \\
 $2^{14}-3\phantom{0} = 16381$   & 572            \\
 $2^{15}-19 = 32749$  & 219            \\
 $2^{16}-15 = 65521$  & 17364          \\
 $2^{17}-1\phantom{0} = 131071$  & 43165          \\
 $2^{18}-5\phantom{0} = 262139$  & 92717          \\
 $2^{19}-1\phantom{0} = 524287$  & 283741         \\
 $2^{20}-3\phantom{0} = 1048573$ & 380985         \\
 $2^{21}-9\phantom{0} = 2097143$  & 360889         \\
 $2^{22}-3\phantom{0} = 4194301$  & 914334         \\
 $2^{23}-15 = 8388593$  & 653276        \\
 $2^{24}-3\phantom{0} = 16777213$  & 6423135       \\
 $2^{25}-39 = 33554393$  & 25907312     \\
 $2^{26}-5\phantom{0} = 67108859$  & 26590841      \\
 $2^{27}-39 = 134217689$  & 45576512     \\
 $2^{28}-57 = 268435399$  & 31792125     \\
 $2^{29}-3\phantom{0} = 536870909$   & 16538103     \\
 $2^{30}-35 = 1073741789$  & 5122456     \\
\hline
\end {tabular}
\end {center}
\end {table}


%%%%%%%%%%%%%%%%%%%%%%%%%%%%%%%%%%%%%%%%%%%%%%%%%%%%%%%%%%%%%%%


\setbox0=\vbox {\hsize = 6.1in
\smallc
\verbatiminput{../examples/fcoll.c}
}

\begin{figure}
\centering
\myboxit{\box0}
\caption {Applying the collision test to a family of LCGs.}
\label{fig:tmultin}
\end{figure}



%%%%%%%%%%%%%%%%%%%%%%%%%%%%%%%%%%%%%%%%%%%%%%%%%%%%%%%%%%%%%%%
\paragraph*{An example: The collision test applied to a family of LCG's.} \



For a concrete illustration, Figure~\ref{fig:tmultin} shows 
a program applying the collision test (see tests {\tt sknuth\_Collision}
and {\tt smultin\_Multinomial})
systematically to a family of LCGs. First, the call to
 {\tt fcong\_CreateLCG} creates the family of generators to which
the tests are applied; i.e., each generator
instance has its  parameters predefined in the file {\tt LCGGood.par}.
These  ``good LCGs'' have prime modulus $m$ (the largest prime less
 than $2^i$),  full period length $m-1$,
and perform well in the spectral test for up to dimension 8.
They are taken from \cite{rLEC99c} and are listed in 
Table~\ref{tab:lcg1} for $10\le i\le 30$.


The following instruction in the program, {\tt smultin\_CreateParam} with
 {\tt NbDelta = 1} and {\tt ValDelta[0] = -1}, specifies that only
the collision test will be applied. The {\tt par} parameter in the test
 could be replaced by the  {\tt NULL} pointer,
in which case default values would be used and it would not be necessary to
 create a  {\tt par} structure. The next instruction {\tt fmultin\_CreateRes}
 creates a structure to hold the results of the test. If one does not want
 to postprocess the results after all the tests are ended, one
may also pass a  {\tt NULL} pointer as argument {\tt res} to the test, in
which case it would not be necessary to create a  {\tt res} structure
either. The instruction  {\tt fcho\_CreateSampleSize} 
specifies that the sample size $n$ (the number of points) 
will be chosen as $n = 2^{i/2 + j}$. 
The next instruction {\tt fmultin\_CreatePer\_DT} says that 
the number of cells $k$ will be equal to the period length of the 
generator tested, and that the relation between $k$, the one-dimensional
interval $d$, and the dimension  $t$ (here 2) is given by $k=d^t$.
Then these two ``choose'' functions are set in the structure {\tt cho}
that will be passed as argument to the test.
The call {\tt fmultin\_Serial1} launches the series of collision
tests, with the parameters $N=1$, $r=0$, $t=2$, 
{\tt Sparse = TRUE}.  LCGs with period lengths near $2^i$ will
be tested for $i=10,11,\dots,30$, each with sample size $n = 2^{i/2 + j}$
for $j=1,2,3,4,5$. Finally, all the created structures are deleted to
free the memory used by each of them.



\begin {table}
\centering
\caption {Expected number of
          collisions and observed number of collisions.}
\label {tab:coll1}
\smallskip

%%%%%%%%%%%%%%%%%%%%%
\begin {tabular}{|c|@{\extracolsep{10pt}}rr|rr|rr|rr|rr|}
%% \multicolumn{11}{l}{\makebox[0pt][l]{fmultin\_Serial1: Collision test, Expected numbers---fmultin\_Serial1: Collision test, Observed numbers}}\\
\hline
 LSize& \multicolumn{2}{c|}{$  j=1 $} & \multicolumn{2}{c|}{$  j=2 $} & \multicolumn{2}{c|}{$  j=3 $} & \multicolumn{2}{c|}{$  j=4 $} & \multicolumn{2}{c|}{$  j=5 $}  \\
\hline
 10 &     1.93 &        2 &     7.62 &        2 &    29.39 &       11 &   108.94 &       58 &   376.52 &      570 \\
 11 &     1.95 &        1 &     7.81 &        0 &    30.44 &        6 &   115.16 &       17 &   413.38 &      474 \\
 12 &     1.96 &        0 &     7.81 &        2 &    30.65 &       13 &   117.87 &       44 &   436.20 &      216 \\
 13 &     2.00 &        0 &     7.95 &        2 &    31.37 &       13 &   121.97 &       55 &   461.02 &      211 \\
 14 &     1.98 &        1 &     7.90 &        5 &    31.31 &       11 &   122.77 &       52 &   471.77 &      191 \\
 15 &     1.99 &        1 &     7.93 &        1 &    31.51 &        7 &   124.27 &       41 &   483.04 &      132 \\
 16 &     1.99 &        1 &     7.95 &        0 &    31.65 &        5 &   125.35 &       20 &   491.26 &       75 \\
 17 &     1.99 &        0 &     7.97 &        1 &    31.75 &       11 &   126.11 &       23 &   497.29 &      101 \\
 18 &     2.00 &        0 &     7.98 &        4 &    31.83 &       15 &   126.66 &       45 &   501.47 &      214 \\
 19 &     2.00 &        0 &     7.98 &        1 &    31.88 &        3 &   127.07 &       19 &   504.60 &       61 \\
 20 &     2.00 &        0 &     7.99 &        0 &    31.91 &        4 &   127.33 &       14 &   506.69 &       74 \\
 21 &     2.00 &        0 &     7.99 &        1 &    31.94 &       12 &   127.55 &       48 &   508.33 &      178 \\
 22 &     2.00 &        0 &     7.99 &        3 &    31.96 &       14 &   127.66 &       59 &   509.34 &      220 \\
 23 &     2.00 &        0 &     8.00 &        0 &    31.97 &        5 &   127.78 &       16 &   510.21 &       45 \\
 24 &     2.00 &        0 &     8.00 &        0 &    31.98 &        1 &   127.83 &       15 &   510.67 &       37 \\
 25 &     2.00 &        1 &     8.00 &        4 &    31.99 &        8 &   127.91 &       44 &   511.16 &      142 \\
 26 &     2.00 &        2 &     8.00 &        1 &    31.99 &        7 &   127.92 &       24 &   511.33 &       71 \\
 27 &     2.00 &        1 &     8.00 &        1 &    31.99 &        7 &   127.94 &       38 &   511.55 &      152 \\
 28 &     2.00 &        0 &     8.00 &        3 &    31.99 &       17 &   127.96 &       41 &   511.67 &      193 \\
 29 &     2.00 &        1 &     8.00 &        0 &    32.00 &        0 &   127.98 &        8 &   511.78 &       25 \\
 30 &     2.00 &        0 &     8.00 &        3 &    32.00 &       10 &   127.98 &       29 &   511.83 &      189 \\
\hline
\end {tabular} \\
\medskip
\end {table}


Tables~\ref{tab:coll1} and \ref{tab:coll11} give the results
of this program.
In Table~\ref{tab:coll1}, for each value of $i$ and $j$, the expected
number of collisions is given on the left, and the observed number of
collisions on the right.
We see that for $j\ge 3$, the observed number of collisions is generally
much too small (there are exceptions at $j=5$ and $i =$ 10 and 11).
The $p$-values written in Table~\ref{tab:coll11} are those which fall
outside the interval $[0.01, 0.99]$, which may be called suspect $p$-values.
A $p$-value smaller than $10^{-300}$ is noted by $\epsilon$.
A $p$-value larger than $1 - 10^{-15}$ is noted by $-\epsilon_1$
($p$-values close to 1 are written as $-p$ instead of $1-p$).
We see that for $j\ge 3$, the $p$-values in Table~\ref{tab:coll11}
are very close to 1.
This means that the two-dimensional points produced by these generators
are too evenly distributed and the test starts detecting this when the
sample size reaches $n_0 \approx 2^{i/2 + 3} \approx 8 \sqrt{\rho}$.
Very clear rejection occurs in all cases for $j=4$, i.e., at sample size
$n \approx 16 \sqrt{\rho}$.




\begin {table}
\centering
\caption {The $p$-values of the collision test for the good LCGs, 
          for $t=2$ and $k\approx 2^i$.}
\label {tab:coll11}
\smallskip
Total CPU time: 00:00:06.69\\[4pt]

\begin {tabular}{|c|@{\extracolsep{10pt}}ccccc|}
% \multicolumn{6}{l}{\makebox[0pt][l]{fmultin\_Serial1: Collision test, p-value for discrete statistic}}\\
\hline
LSize & $ j= 1 $ & $ j= 2 $ & $ j= 3 $ & $ j= 4 $ & $ j= 5$  \\
\hline
 10   &   &          &  --1.0e--5\phantom{0} & --1.1e--12 &   \eps   \\
 11   &   &  --2.5e--4 &  --6.3e--9\phantom{0}  &  \epsm   &   1.1e--6 \\
 12   &   &          &  --1.1e--4\phantom{0}  &  \epsm   &  \epsm   \\
 13   &   &          &  --9.0e--5\phantom{0}  & --6.0e--14 &  \epsm   \\
 14   &   &          &  --1.5e--5\phantom{0}  & --9.4e--15 &  \epsm   \\
 15   &   &  --2.9e--3 &  --9.0e--8\phantom{0}  &  \epsm   &  \epsm   \\
 16   &   &  --3.2e--4 &  --3.5e--9\phantom{0}  &  \epsm   &  \epsm   \\
 17   &   &  --3.0e--3 &  --1.6e--5\phantom{0}  &  \epsm   &  \epsm   \\
 18   &   &          &  --6.6e--4\phantom{0}  &  \epsm   &  \epsm   \\
 19   &   &  --3.0e--3 & --7.0e--11 &  \epsm   &  \epsm   \\
 20   &   &  --3.3e--4 & --6.0e--10 &  \epsm   &  \epsm   \\
 21   &   &  --3.0e--3 &  --4.7e--5\phantom{0}  &  \epsm   &  \epsm   \\
 22   &   &          &  --2.9e--4\phantom{0}  & --7.1e--12 &  \epsm   \\
 23   &   &  --3.3e--4 &  --4.1e--9\phantom{0}  &  \epsm   &  \epsm   \\
 24   &   &  --3.3e--4 & --4.1e--13 &  \epsm   &  \epsm   \\
 25   &   &          &  --4.5e--7\phantom{0}  &  \epsm   &  \epsm   \\
 26   &   &  --3.0e--3 &  --1.1e--7\phantom{0}  &  \epsm   &  \epsm   \\
 27   &   &  --3.0e--3 &  --1.1e--7\phantom{0}  &  \epsm   &  \epsm   \\
 28   &   &          &  --2.8e--3\phantom{0}  &  \epsm   &  \epsm   \\
 29   &   &  --3.4e--4 & --1.3e--14 &  \epsm   &  \epsm   \\
 30   &   &          &  --5.6e--6\phantom{0}  &  \epsm   &  \epsm   \\
\hline
\end {tabular} \\
\medskip

\end {table}


%%%%%%%%%%%%%%%%%%%%%%%%%%%%%%%%%%%%%%%%%%%%%%%%%%%%%%%%%%%%%%%

\paragraph*{Another example: The {birthday spacings} test applied
  to a family of LCG's.} \

The next example in Figure \ref{tab:fbirth} gives a program applying the 
{birthday spacings} test (see {\tt smarsa\_BirthdaySpacings}) to a
family of generators. 
First the family {\tt LCGPow2} is created by calling
function {\tt fcong\_CreateLCGPow2}. Each generator of the family
has its parameters predefined in file {\tt LCGPow2.par}. These generators 
have been chosen such that their modulus $m$ is exactly equal to a power of 2,
i.e. $m = 2^i$ for $10 \le i \le 30$, and their multiplier
gives them a good structure according to the spectral test.
The sample size of the test (the number of points $n$) is then chosen 
(by calling {\tt fcho\_CreateSampleSize}) to follow the law $n = 2^{i/3 + j}$, 
for generator of modulus $m=2^i$. 
The next instruction {\tt fmarsa\_CreateBirthEC} indicates that, given
$N=1$ replication of the test and dimension $t=2$, the number of segments $d$
on the interval $[0, 1)$ will be chosen so that the expected
number of collisions is (approximately) equal to 1.
These two ``choose'' functions are set in the structure {\tt cho} by the call
{\tt fcho\_CreateCho2} and will thus be
 passed as arguments to the tests.
Then function {\tt fmarsa\_BirthdayS1} applies the birthday spacings test
on the 21 selected generators of the family, for $1 \le j \le 5$ and with the 
other parameters determined by the above functions. After all the tests are
completed, the following {\tt Delete} functions free the resources used
by the program.


\setbox0=\vbox {\hsize = 6.1in
\smallc
\verbatiminput{../examples/fbirth.c}
}

\begin{figure}
\centering
\myboxit{\box0}
\caption {Applying the birthday spacings test to a family of LCGs.}
\label{tab:fbirth}
\end{figure}



Tables~\ref{tab:birth.res1} and \ref{tab:birth.res2}
give the results of this program.
In Table~\ref{tab:birth.res1}, for each value of $i$ and $j$, the expected
number of collisions is given on the left and the observed number of
collisions on the right.
We see that for $j\ge 2$, the observed number of collisions is
much too large and the more so as $j$ increases.
The right $p$-values written in Table~\ref{tab:birth.res2} are those which fall
outside the interval $[0.01, 0.99]$. A $p$-value smaller than $10^{-300}$ 
is noted by $\epsilon$. For $j = 1$, the number of collisions is close 
to the expected value and the corresponding $p$-values are in the interval
$[0.01, 0.99]$. The generators pass the test also for $j \le 1$.
But for $j\ge 2$, the $p$-values in Table~\ref{tab:birth.res2}
becomes smaller as $j$ increases.
The tests signals catastrophic failures of the generators already for $j\ge3$.
This is because the two-dimensional points produced by these generators
are too evenly distributed and the test starts detecting this when the
sample size reaches $n \approx 2^{i/3 + 2}$. % \approx 16 \sqrt{\rho}$.
These are quite small sample sizes; for example, for the generator
with $m= 2^{30}$, the tests start to fail for $n$ as small as 4096.


\begin {table}
\centering
\caption {Birthday spacings test: Expected and 
  observed number of collisions for each $i$ and $j$.}
\label {tab:birth.res1}
\smallskip
%%%%%%%%%%%%%%%%%%%%%
\begin {tabular}{|c|@{\extracolsep{10pt}}rr|rr|rr|rr|rr|}
\hline
 LSize& \multicolumn{2}{c|}{$  j=1 $} & \multicolumn{2}{c|}{$  j=2 $} & \multicolumn{2}{c|}{$  j=3 $} & \multicolumn{2}{c|}{$  j=4 $} & \multicolumn{2}{c|}{$  j=5 $}  \\
\hline
 10 &   ---    &   ---    &     1.01 &        5 &     1.00 &       48 &     1.00 &      131 &   ---    &   ---    \\
 11 &   ---    &   ---    &     1.01 &        8 &     1.00 &       52 &     1.00 &      147 &     1.00 &      377 \\
 12 &     1.01 &        0 &     1.00 &       10 &     1.00 &       44 &     1.00 &      204 &     1.00 &      447 \\
 13 &     1.01 &        0 &     1.00 &       11 &     1.00 &       50 &     1.00 &      196 &     1.00 &      579 \\
 14 &     1.01 &        0 &     1.00 &       10 &     1.00 &       56 &     1.00 &      247 &     1.00 &      662 \\
 15 &     1.00 &        2 &     1.00 &        9 &     1.00 &       55 &     1.00 &      258 &     1.00 &      911 \\
 16 &     1.00 &        1 &     1.00 &        5 &     1.00 &       61 &     1.00 &      316 &     1.00 &      958 \\
 17 &     1.00 &        0 &     1.00 &        7 &     1.00 &       70 &     1.00 &      365 &     1.00 &     1172 \\
 18 &     1.00 &        1 &     1.00 &       10 &     1.00 &       73 &     1.00 &      385 &     1.00 &     1374 \\
 19 &     1.00 &        4 &     1.00 &       13 &     1.00 &       79 &     1.00 &      414 &     1.00 &     1600 \\
 20 &     1.00 &        0 &     1.00 &        7 &     1.00 &       72 &     1.00 &      449 &     1.00 &     1895 \\
 21 &     1.00 &        1 &     1.00 &        5 &     1.00 &       73 &     1.00 &      509 &     1.00 &     2176 \\
 22 &     1.00 &        4 &     1.00 &        8 &     1.00 &       78 &     1.00 &      505 &     1.00 &     2472 \\
 23 &     1.00 &        3 &     1.00 &       13 &     1.00 &       79 &     1.00 &      522 &     1.00 &     2797 \\
 24 &     1.00 &        1 &     1.00 &       16 &     1.00 &       59 &     1.00 &      580 &     1.00 &     3092 \\
 25 &     1.00 &        2 &     1.00 &        9 &     1.00 &      102 &     1.00 &      588 &     1.00 &     3392 \\
 26 &     1.00 &        0 &     1.00 &       10 &     1.00 &       79 &     1.00 &      629 &     1.00 &     3709 \\
 27 &     1.00 &        2 &     1.00 &       11 &     1.00 &       89 &     1.00 &      636 &     1.00 &     3915 \\
 28 &     1.00 &        1 &     1.00 &        8 &     1.00 &       83 &     1.00 &      691 &     1.00 &     4097 \\
 29 &     1.00 &        2 &     1.00 &       12 &     1.00 &       89 &     1.00 &      650 &     1.00 &     4500 \\
 30 &     1.00 &        3 &     1.00 &       16 &     1.00 &       80 &     1.00 &      649 &     1.00 &     4551 \\
\hline
\end {tabular} \\
\medskip

\end {table}


\begin {table}
\centering
\caption {The right $p$-values $P[Y \ge y]$ of the BirthdaySpacings test
 for the LCGPow2s.}
\label {tab:birth.res2}
\smallskip
Total CPU time: 00:00:00.07\\[4pt]

\begin {tabular}{|c|@{\extracolsep{10pt}}ccccc|}
\hline
LSize & $ j= 1 $ & $ j= 2 $ & $ j= 3 $ & $ j= 4 $ & $ j= 5$  \\
\hline
 10   & --- &   3.8e--3\phantom{0} &  3.7e--62\phantom{0} & 4.9e--223 &   ---    \\
 11   & --- &   1.1e--5\phantom{0} &  5.2e--69\phantom{0} & 2.2e--257 &   \eps   \\
 12   &     &   1.1e--7\phantom{0} &  1.4e--55\phantom{0} &   \eps   &   \eps   \\
 13   &     &   1.0e--8\phantom{0} &  1.3e--65\phantom{0} &   \eps   &   \eps   \\
 14   &     &   1.1e--7\phantom{0} &  5.3e--76\phantom{0} &   \eps   &   \eps   \\
 15   &     &   1.1e--6\phantom{0} &  3.0e--74\phantom{0} &   \eps   &   \eps   \\
 16   &     &   3.7e--3\phantom{0} &  7.4e--85\phantom{0} &   \eps   &   \eps   \\
 17   &     &   8.3e--5\phantom{0} & 3.2e--101            &   \eps   &   \eps   \\
 18   &     &   1.1e--7\phantom{0} & 8.5e--107            &   \eps   &   \eps   \\
 19   &     &  6.4e--11            & 4.2e--118            &   \eps   &   \eps   \\
 20   &     &   8.3e--5\phantom{0} & 6.1e--105            &   \eps   &   \eps   \\
 21   &     &   3.7e--3\phantom{0} & 8.3e--107            &   \eps   &   \eps   \\
 22   &     &   1.0e--5\phantom{0} & 3.3e--116            &   \eps   &   \eps   \\
 23   &     &  6.4e--11            & 4.2e--118            &   \eps   &   \eps   \\
 24   &     &  1.9e--14            &  2.7e--81\phantom{0} &   \eps   &   \eps   \\
 25   &     &   1.1e--6\phantom{0} & 3.9e--163            &   \eps   &   \eps   \\
 26   &     &   1.1e--7\phantom{0} & 4.2e--118            &   \eps   &   \eps   \\
 27   &     &   1.0e--8\phantom{0} & 2.3e--137            &   \eps   &   \eps   \\
 28   &     &   1.0e--5\phantom{0} & 9.4e--126            &   \eps   &   \eps   \\
 29   &     &  8.3e--10            & 2.3e--137            &   \eps   &   \eps   \\
 30   &     &  1.9e--14            & 5.2e--120            &   \eps   &   \eps   \\
\hline
\end {tabular} \\
\medskip
\end {table}




\iffalse   %%%%%%%%%%%%%%%%%%%%%%%%%%%%%%%%%%%%


\begin {table}
\begin {center}
\caption {Some good and bad (baby) LCGs}
\label {tab:lcg1t}
\smallskip 
\begin {tabular}{|lrr|}
\hline
\qquad $m$             & $a$ (good LCG)  & $a$ (bad LCG)  \\
\hline
 $2^{10}-3  = 1021$    &  65             & 127  \\
 $2^{11}-9  = 2039$    & 995             & 1359 \\
 $2^{12}-3 = 4093$    & 209             & 5  \\
 $2^{13}-1 = 8191$    & 884             & 2341 \\
 $2^{14}-3 = 16381$   & 572             & 2731 \\
 $2^{15}-19 = 32749$  & 219             & 10 \\
 $2^{16}-15 = 65521$  & 17364           & 17 \\
 $2^{17}-1 = 131071$  & 43165           & 68985 \\
 $2^{18}-5 = 262139$  & 92717           & 203883 \\
 $2^{19}-1 = 524287$  & 283741          & 458756 \\
 $2^{20}-3 = 1048573$ & 380985          & 213598 \\
 $2^{21}-9 = 2097143$  & 360889          & 202947 \\
 $2^{22}-3 = 4194301$  & 914334          & 4079911 \\
 $2^{23}-15 = 8388593$  & 653276         & 2696339 \\
 $2^{24}-3 = 16777213$  & 6423135        & 486293 \\
 $2^{25}-39 = 33554393$  & 25907312      & 5431467 \\
 $2^{26}-5 = 67108859$  & 26590841       & 42038579 \\
 $2^{27}-39 = 134217689$  & 45576512     & 24990322 \\
 $2^{28}-57 = 268435399$  & 31792125     & 31842465 \\
 $2^{29}-3 = 536870909$   & 16538103     & 8903330 \\
 $2^{30}-35 = 1073741789$  & 5122456     & 3930720 \\
 $2^{31}-1 = 2147483647$  &  1389796     & 868723 \\
 $2^{32} -5 = 4294967291$ & 1588635695   & 1123161 \\
 $2^{33}-9 = 8589934583 $ & 7425194315   & 1026767 \\
 $2^{34}-41 = 17179869143$ & 5295517759  & 10045 \\
 $2^{35}-31 = 34359738337$ & 3124199165  & 10052 \\
 $2^{36}-5 = 68719476731$  & 49865143810 & 102254510 \\
% $2^{37}-25 = 137438953447$ & 76886758244 & 87666368 \\
% $2^{38}-45  $              & 17838542566 & 1045020214 \\
% $2^{39}-7   $              & 61992693052 & 809807353 \\
% $2^{40}-87  $            & 1038914804222 & 87666376 \\
\hline
\end {tabular}
\end {center}
\end {table}


We apply a test to a whole family while varying the sample size $n$
following a law of the form $n=2^{\alpha\, e + \beta}$. If we happen to
select suitable $\alpha$ and $\beta$, we may be able to discover a
 regularity between the sample size and the size (modulus) of the
generator.


Par exemple, la table \ref{tab:coll} montre les $p$-values du test
de Collisions pour les
{\tt BadLCG2} en fonction de $n$, pour un nombre de cases $k\approx 2^e$
en 2 dimensions, et o\`u $\epsilon$ denote une value $< 10^{-15}$.
 Les values de $n$ sont donnees par
$$
n = 2^{\frac e 2 + \nu}.
$$


\begin {table}
\centering
\caption {The $p$-values of $C$ for the bad LCGs, for $t=2$ 
and $k\approx 2^e$.}
\label {tab:coll}
\smallskip
\begin {tabular}{|r|@{\extracolsep{16pt}}lllll|}
\hline
 $i$& $\nu=-3$ & $\nu=-2$ & $\nu=-1$ & $\nu=0$ & $\nu=1$  \\
\hline
 14 &          & 4.5E-4   & 7.0E-3   & 3.1E-9  & \eps     \\
 15 &          &          & 9.2E-6   & 1.6E-10 & \eps     \\
 16 &          & 4.6E-4   & 7.1E-3   & 6.0E-8  & \eps     \\
 17 &          & 4.8E-4   &          & 6.0E-8  & \eps     \\
 18 &          &          & 7.1E-3   & 3.4E-9  & \eps     \\
 19 &          & 4.7E-4   &          & 3.4E-9  & \eps     \\
 20 &          &          &          & 1.7E-4  & \eps     \\
 21 &          &          &          & 3.4E-9  & \eps     \\
 22 &          & 4.8E-4   & 7.2E-3   & 3.2E-13 & \eps     \\
 23 &          & 4.8E-4   &          & 1.7E-4  & \eps     \\
 24 &          &          & 7.2E-3   & 1.7E-4  & \eps     \\
 25 &          &          & 3.0E-4   & 1.0E-6  & \eps     \\
 26 &          &          &          & 1.0E-6  & \eps     \\
 27 &          &          & 7.2E-3   & 1.4E-5  & \eps     \\
 28 &          &          & 7.2E-3   & 6.2E-8  & \eps     \\
 29 &          & 4.8E-4   & 7.2E-3   & 1.0E-6  & \eps     \\
 30 &          &          & 3.0E-4   & \eps    & \eps     \\
 31 &          & 4.8E-4   & 7.2E-3   & 1.8E-3  & \eps     \\
 32 &          &          & 9.2E-6   & 6.2E-8  & \eps     \\
 33 &          & 4.8E-4   & 7.2E-3   & 1.8E-3  & 2.4E-4 \\
 34 &          & 4.8E-4   & 9.2E-6   &         & \eps     \\
\hline
\end {tabular}
\end {table}


\fi   %%%%%%%%%%%%%%%%%%%%%%%%%%%%%%%%%%%%%%%%%%%%%%%%%%
